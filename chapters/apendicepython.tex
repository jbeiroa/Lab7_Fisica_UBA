\chapter{Programa de adquisición y control para estudios comportamentales}
\label{ap:programapython}

El programa está íntegramente escrito en Python 3.5 y utiliza las librerías Sound Device y PyAudio para grabar y emitir sonidos respectivamente. Cabe aclarar que se puede ``grabar'' cualquier señal analógica de frecuencia superior a 20Hz con una amplitud pico a pico de alrededor de 200mV a 1V. Al no ser datos requeridos por los consumidores, los fabricantes rara vez los listan en las hojas de datos. Por ese motivo, es recomendable aumentar el valor pico a pico de tensión de la señal a medir hasta observar que la PC satura (el efecto se lo llama \emph{clipping}) la señal.

Este programa se utilizó para grabar en estéreo las señales del par de electrodos y la del hidrófono. La segunda sirve como una ``estampa temporal'', permite discernir en qué momento se emitió el estímulo.

Para terminar, utilizando distintas teclas del teclado uno puede abrir archivos de texto en los cuales se guardarán las mediciones y emitir sonidos desde un archivo (un .wav) o tonos puros de frecuencia, amplitud y duración programada por el usuario en el momento.

\section{Programa principal}
\lstinputlisting[language=Python]{chapters/code/experiment.py}
\section{Reproductor de sonidos}
\lstinputlisting[language=Python]{chapters/code/ToneGenerator.py}
\section{Controlador Arduino}
\lstinputlisting[language=Python]{chapters/code/arduino.py}
