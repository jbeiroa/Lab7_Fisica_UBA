\chapter{Conclusiones}

Los principales objetivos planteados en el plan de trabajo han sido cumplidos: la caracterización de la pecera para estudios comportamentales permitió demarcar una zona en la cual realizar los experimentos. Las principales complicaciones provinieron de distintas obras y cortes de luz en los laboratorios de la Facultad y de los altos costos de equipos como el acelerómetro.

Las mediciones de la RI permiten saber en qué zonas de la pecera se excitan modos normales y los modelos numéricos cómo son dichos modos. Los resultados obtenidos para bajas frecuencias obligan a la elección de una condición de impedancia por sobre la de bordes suaves en la superficie libre. Los modelos realizados co ndicha condición muestran un buen grado de acuerdo con las mediciones realizadas. Al utilizar la aproximación de bordes suaves en todas las superficies el acuario se comporta como una guía de ondas con sus primeros modos por arriba de 6kHz, todos presentes en las mediciones, pero no se predicen los modos de baja frecuencia.

La inclusión del neoprene mostró eliminar todos los modos de alta frecuencia y disminuir la intensidad de los modos más bajos. A futuro sería interesante y útil elaborar algún método que permita también eliminar los modos de bajas frecuencias y así tener una respuesta plana en toda la pecera y a todas las frecuencias de interés.

En cuanto a los desarrollos puramente técnicos se ha avanzado en la automatización de los experimentos del laboratorio. El programa de adquisición y control realizado se utilizará tanto en estudios de peces eléctricos como dorados y puede ser modificado y ampliado sin dificultades mayores según se requiera en el futuro.